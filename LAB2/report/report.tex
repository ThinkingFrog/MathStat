\documentclass[12pt,a4paper]{article}

\usepackage[T1,T2A]{fontenc}
\usepackage[utf8]{inputenc}
\usepackage[english, russian]{babel}
\usepackage{indentfirst}
\usepackage{misccorr}
\usepackage{graphicx}
\usepackage{amsmath}
\usepackage{graphicx}
\usepackage{float}
\usepackage[left=20mm,right=10mm, top=20mm,bottom=20mm,bindingoffset=0mm]{geometry}

\setlength{\parskip}{6pt}\graphicspath{{images/}}\DeclareGraphicsExtensions{.png}

\begin{document}

    \begin{titlepage}
        \begin{center}
            \large
            Санкт-Петербургский политехнический университет\\Петра Великого\\
            \vspace{0.5cm}
            Институт прикладной математики и механики\\
            \vspace{0.25cm}
            Кафедра «Прикладная математика»
            \vfill
            \textsc{\LARGE\textbf{Отчет по лабораторной работе №2}}\\[5mm]
            \Large
            по дисциплине\\"Математическая статистика"
        \end{center}
        \vfill
        \begin{tabular}{l p{175pt} l}
            Выполнил студент \\ группы 3630102/80201 && Хрипунков Дмитрий Викторович
            \vspace{0.25cm}
            \\Проверил \\ доцент, к.ф.-м.н. && Баженов Александр Николаевич
        \end{tabular}
        \vfill
        \begin{center}
            Санкт-Петербург \\ 2021 г.
        \end{center}
    \end{titlepage}

\newpage
\begin{center}
    \tableofcontents
    \setcounter{page}{2}
\end{center}
\newpage
\begin{center}
    \listoftables
\end{center}

\newpage
\section{Постановка задачи}
Для 5 распределений:
\begin{itemize}
    \item Нормальное распределение $N(x,0,1)$
    \item Распределение Коши $C(x,0,1)$
    \item Распределение Лапласа $L(x,0,\frac{1}{\sqrt{2}})$
    \item Распределение Пуассона $P(k,10)$
    \item Равномерное распределение $U(x,-\sqrt{3},\sqrt{3})$
\end{itemize}

Необходимо:
\begin{enumerate}
    \item Сгенерировать выборки размером 10, 100 и 1000 элементов
    \item Вычислить для каждой них статистические характеристики положения данных: \\ $\overline{x}, med x, z_R, z_Q, z_tr$
    \item Повторить данные вычисления 1000 раз для каждой выборки и найти среднее характеристик положения $E(z)=\overline{z}$ и вычислить оценку дисперсии $D(z)=\overline{z^2}-{\overline{z}}^2$
    \item Представить полученные результаты в виде таблиц
\end{enumerate}

\section{Теория}
\subsection{Рассматриваемые распределения}
Плотности:
\begin{itemize}
		\item Нормальное распределение
		    \begin{equation}
			    N(x,0,1)=\frac{1}{\sqrt{2\pi}}e^{-\frac{x^2}{2}}
			    \label{normal} 
			\end{equation}
		\item Распределение Коши
		    \begin{equation}
				C(x,0,1)=\frac{1}{\pi}\frac{1}{x^2+1}
				\label{cauchy}
			\end{equation} 
		\item Распределение Лапласа
		    \begin{equation}
				L(x,0,\frac{1}{\sqrt{2}})=\frac{1}{\sqrt{2}}e^{-\sqrt{2}|x|}
				\label{laplace} 
			\end{equation}
		\item Распределение Пуассона
		    \begin{equation}
				P(k,10)=\frac{10^k}{k!}e^{-10}
				\label{poisson}
			\end{equation}
		\item Равномерное распределение
		    \begin{equation}
				U(x,-\sqrt{3},\sqrt{3})=
				\begin{cases}
					\frac{1}{2\sqrt{3}},|x|\leq\sqrt{3}\\0,|x|>\sqrt{3}
				\end{cases}
				\label{uniform}
			\end{equation}
\end{itemize}

\subsection{Выборочные числовые характеристики}
\textit{Вариационный ряд} - последовательность элементов выборки, расположенных в неубывающем порядке.

\subsubsection{Характеристики положения}
\begin{itemize}
    \item Выборочное среднее
        \begin{equation}
            \overline{x}=\frac{1}{n}\sum_{i=1}^{n}{x_i}
		\end{equation}
	\item Выборочная медиана
	    \begin{equation}
			med x=
			\begin{cases}
			    x_{l+1},n=2l+1\\
				\frac{x_l+x_{l+1}}{2},n=2l
			\end{cases}
		\end{equation}
	\item Полусумма экстремальных выборочных элементов
	    \begin{equation}
			z_R=\frac{x_1 + x_n}{2}
		\end{equation}
	\item Полусумма квартилей
	    \newline Выборочная квартиль $z_p$ порядка $p$ определяется формулой
	    \begin{equation}
		    z_p =
			\begin{cases}
			    x_{[np]+1},np-\text{дробное}\\
		      	x_{np},np-\text{целое}
	        \end{cases}
		\end{equation}
	    Полусумма квартилей
	    \begin{equation}
			z_Q=\frac{z_{1/4}+z_{3/4}}{2}
		\end{equation}
	\item Усечённое среднее
	    \begin{equation}
			z_{tr}=\frac{1}{n-2r}\sum_{i=r+1}^{n-r}{x_i}, r\approx\frac{n}{4}
		\end{equation}
\end{itemize}

\subsubsection{Характеристики рассеивания}
Выборочная дисперсия определяется по формуле:
\begin{equation}
    D=\frac{1}{n}\sum^{n}_{i=1}{(x_i-\overline{x})^2}
\end{equation}

\section{Реализация}
Лабораторная работа выполнена на языке Python в виртуальной среде Anaconda с интерпретатором версии 3.9 в среде разработки Visual Studio Code. Дополнительные зависимости:
\begin{itemize}
    \item scipy
    \item numpy
\end{itemize}

Исходный код размещён в git-репозитории на GitHub: \\ https://github.com/ThinkingFrog/MathStat

\section {Результаты}
\begin{table}[H]
    \centering
    \begin{tabular}{|l||c|c|c|c|c|}
        \hline
        & $\overline{x}$ & $med(x)$ & $z_R$ & $z_Q$ & $z_{tr}$\\\hline\hline
        size 10 & & & & &\\\hline
        $E(z)$ & -0.008474 & -0.011502 & -0.004567 & 0.313433 & 0.26929 \\\hline
        $D(z)$ & 0.092723 & 0.133581 & 0.185508 & 0.117345 & 0.105668 \\\hline
        $E + \sqrt D$ & 0.296032 & 0.353986 & 0.426139 & 0.655989 & 0.594356 \\\hline
        $E - \sqrt D$ & -0.312979 & -0.37699 & -0.435273 & -0.029124 & -0.055777 \\\hline
        Estimation & 0 & 0 & 0 & 0 & 0 \\\hline
        size 100 & & & & &\\\hline
        $E(z)$ & 0.001189 & 0.001817 & 0.001457 & 0.016332 & 0.030168 \\\hline
        $D(z)$ & 0.010864 & 0.016026 & 0.10168 & 0.013248 & 0.012352 \\\hline
        $E + \sqrt D$ & 0.105419 & 0.12841 & 0.32033 & 0.131432 & 0.141308 \\\hline
        $E - \sqrt D$ & -0.10304 & -0.124777 & -0.317415 & -0.098768 & -0.080973 \\\hline
        Estimation & 0 & 0 & 0 & 0 & 0 \\\hline
        size 1000 & & & & &\\\hline
        $E(z)$ & -0.000752 & -0.000457 & -0.008089 & 0.000512 & 0.00199 \\\hline
        $D(z)$ & 0.001038 & 0.001648 & 0.060405 & 0.001275 & 0.001246 \\\hline
        $E + \sqrt D$ & 0.031466 & 0.040137 & 0.237685 & 0.036224 & 0.037286 \\\hline
        $E - \sqrt D$ & -0.03297 & -0.041052 & -0.253863 & -0.0352 & -0.033306 \\\hline
        Estimation & 0.0 & 0.0 & 0 & 0.0 & 0.0 \\\hline
    \end{tabular}
    \caption{Нормальное распределение \eqref{normal}}
    \label{tab:normal}
\end{table}

\begin{table}[H]
    \centering
    \begin{tabular}{|l||c|c|c|c|c|}
        \hline
        & $\overline{x}$ & $med(x)$ & $z_R$ & $z_Q$ & $z_{tr}$\\\hline\hline
        size 10 & & & & &\\\hline
        $E(z)$ & -0.79855 & -0.00035 & -4.101512 & 1.19138 & 0.698774 \\\hline
        $D(z)$ & 1317.311589 & 0.350235 & 32666.76735 & 15.821066 & 2.479704 \\\hline
        $E + \sqrt D$ & 35.496238 & 0.591457 & 176.637989 & 5.16895 & 2.273481 \\\hline
        $E - \sqrt D$ & -37.093337 & -0.592156 & -184.841013 & -2.78619 & -0.875934 \\\hline
        Estimation & - & 0 & - & 0 & 0 \\\hline
        size 100 & & & & &\\\hline
        $E(z)$ & 1.438919 & 0.007712 & 70.252783 & 0.049879 & 0.049742 \\\hline
        $D(z)$ & 517.855301 & 0.026 & 1282426.145731 & 0.055577 & 0.027417 \\\hline
        $E + \sqrt D$ & 24.195353 & 0.168957 & 1202.69534 & 0.285628 & 0.215322 \\\hline
        $E - \sqrt D$ & -21.317516 & -0.153532 & -1062.189775 & -0.18587 & -0.115837 \\\hline
        Estimation & - & 0 & - & 0 & 0 \\\hline
        size 1000 & & & & &\\\hline
        $E(z)$ & 17.893711 & 0.00024 & 8923.594175 & 0.005589 & 0.006013 \\\hline
        $D(z)$ & 330963.684932 & 0.002206 & 82722241630.99397 & 0.004493 & 0.002395 \\\hline
        $E + \sqrt D$ & 593.188144 & 0.047207 & 296538.33933 & 0.072622 & 0.054951 \\\hline
        $E - \sqrt D$ & -557.400722 & -0.046726 & -278691.150981 & -0.061443 & -0.042925 \\\hline
        Estimation & - & 0.0 & - & 0.0 & 0.0 \\\hline
    \end{tabular}
    \caption{Распределение Коши\eqref{cauchy}}
    \label{tab:cauchy}
\end{table}

\begin{table}[H]
    \centering
    \begin{tabular}{|l||c|c|c|c|c|}
        \hline
        & $\overline{x}$ & $med(x)$ & $z_R$ & $z_Q$ & $z_{tr}$\\\hline\hline
        size 10 & & & & &\\\hline
        $E(z)$ & -0.007129 & 0.004394 & -0.046904 & 0.29745 & 0.235897 \\\hline
        $D(z)$ & 0.097945 & 0.071578 & 0.414221 & 0.113508 & 0.080113 \\\hline
        $E + \sqrt D$ & 0.305832 & 0.271935 & 0.596696 & 0.63436 & 0.51894 \\\hline
        $E - \sqrt D$ & -0.32009 & -0.263147 & -0.690503 & -0.039459 & -0.047146 \\\hline
        Estimation & 0 & 0 & 0 & 0 & 0 \\\hline
        size 100 & & & & &\\\hline
        $E(z)$ & -0.002896 & -0.001324 & 0.003334 & 0.011611 & 0.017572 \\\hline
        $D(z)$ & 0.008925 & 0.005687 & 0.419663 & 0.009482 & 0.005956 \\\hline
        $E + \sqrt D$ & 0.091576 & 0.074088 & 0.651147 & 0.108986 & 0.094748 \\\hline
        $E - \sqrt D$ & -0.097368 & -0.076737 & -0.64448 & -0.085765 & -0.059604 \\\hline
        Estimation & 0.0 & 0.0 & 0 & 0 & 0.0 \\\hline
        size 1000 & & & & &\\\hline
        $E(z)$ & -0.001022 & -5.2e-05 & 0.008371 & 0.001062 & 0.002033 \\\hline
        $D(z)$ & 0.000988 & 0.000519 & 0.386344 & 0.001015 & 0.000612 \\\hline
        $E + \sqrt D$ & 0.030407 & 0.022729 & 0.629937 & 0.032921 & 0.026767 \\\hline
        $E - \sqrt D$ & -0.03245 & -0.022833 & -0.613195 & -0.030797 & -0.0227 \\\hline
        Estimation & 0.0 & 0.0 & 0 & 0.0 & 0.0 \\\hline
    \end{tabular}
    \caption{Распределение Лапласа\eqref{laplace}}
    \label{tab:laplace}
\end{table}

\begin{table}[H]
    \centering
    \begin{tabular}{|l||c|c|c|c|c|}
        \hline
        & $\overline{x}$ & $med(x)$ & $z_R$ & $z_Q$ & $z_{tr}$\\\hline\hline
        size 10 & & & & &\\\hline
        $E(z)$ & 10.0024 & 9.8395 & 10.3185 & 10.937 & 10.766333 \\\hline
        $D(z)$ & 1.083954 & 1.50099 & 1.868308 & 1.477031 & 1.348122 \\\hline
        $E + \sqrt D$ & 11.043531 & 11.064649 & 11.685361 & 12.152332 & 11.92742 \\\hline
        $E - \sqrt D$ & 8.961269 & 8.614351 & 8.951639 & 9.721668 & 9.605247 \\\hline
        Estimation & $10 \pm 1$ & $10 \pm 1$ & $10 \pm 2$ & $10 \pm 2$ & $10 \pm 1$ \\\hline
        size 100 & & & & &\\\hline
        $E(z)$ & 9.9982 & 9.8465 & 10.9 & 9.9605 & 9.94652 \\\hline
        $D(z)$ & 0.100821 & 0.215688 & 1.0055 & 0.16469 & 0.124796 \\\hline
        $E + \sqrt D$ & 10.315724 & 10.310922 & 11.902746 & 10.36632 & 10.299785 \\\hline
        $E - \sqrt D$ & 9.680676 & 9.382078 & 9.897254 & 9.55468 & 9.593255 \\\hline
        Estimation & $10 \pm 1$ & $10 \pm 1$ & $10 \pm 2$ & $10 \pm 2$ & $10 \pm 1$ \\\hline
        size 1000 & & & & &\\\hline
        $E(z)$ & 10.001136 & 9.9965 & 11.6705 & 9.9945 & 9.86845 \\\hline
        $D(z)$ & 0.009839 & 0.003238 & 0.69518 & 0.00422 & 0.011399 \\\hline
        $E + \sqrt D$ & 10.100328 & 10.053401 & 12.504274 & 10.05946 & 9.975218 \\\hline
        $E - \sqrt D$ & 9.901944 & 9.939599 & 10.836726 & 9.92954 & 9.761682 \\\hline
        Estimation & $10 \pm 1$ & $10 \pm 1$ & $10 \pm 2$ & $10 \pm 2$ & $10 \pm 1$ \\\hline
    \end{tabular}
    \caption{Распределение Пуассона\eqref{poisson}}
    \label{tab:poisson}
\end{table}

\begin{table}[H]
    \centering
    \begin{tabular}{|l||c|c|c|c|c|}
        \hline
        & $\overline{x}$ & $med(x)$ & $z_R$ & $z_Q$ & $z_{tr}$\\\hline\hline
        size 10 & & & & &\\\hline
        $E(z)$ & -0.010868 & -0.020561 & -0.001797 & 0.309553 & 0.297099 \\\hline
        $D(z)$ & 0.100284 & 0.229497 & 0.042455 & 0.11875 & 0.152572 \\\hline
        $E + \sqrt D$ & 0.305809 & 0.458498 & 0.20425 & 0.654154 & 0.687703 \\\hline
        $E - \sqrt D$ & -0.327545 & -0.49962 & -0.207843 & -0.035049 & -0.093505 \\\hline
        Estimation & 0 & 0 & 0 & 0 & 0 \\\hline
        size 100 & & & & &\\\hline
        $E(z)$ & -0.000405 & 0.001731 & 0.000333 & 0.016786 & 0.034421 \\\hline
        $D(z)$ & 0.010312 & 0.029023 & 0.000565 & 0.015811 & 0.02071 \\\hline
        $E + \sqrt D$ & 0.101145 & 0.172091 & 0.024098 & 0.142527 & 0.178331 \\\hline
        $E - \sqrt D$ & -0.101954 & -0.16863 & -0.023432 & -0.108955 & -0.109488 \\\hline
        Estimation & 0 & 0 & 0.0 & 0 & 0 \\\hline
        size 1000 & & & & &\\\hline
        $E(z)$ & 0.001586 & 0.002118 & 2.8e-05 & 0.003339 & 0.005729 \\\hline
        $D(z)$ & 0.00095 & 0.002862 & 6e-06 & 0.001452 & 0.001921 \\\hline
        $E + \sqrt D$ & 0.032404 & 0.055621 & 0.002391 & 0.041442 & 0.049564 \\\hline
        $E - \sqrt D$ & -0.029233 & -0.051384 & -0.002335 & -0.034763 & -0.038105 \\\hline
        Estimation & 0.0 & 0.0 & 0.00 & 0.0 & 0.0 \\\hline
    \end{tabular}
    \caption{Равномерное распределение\eqref{uniform}}
    \label{tab:uniform}
\end{table}

\section{Обсуждение}
Полученные данные показывают, что выборки из большего количества элементов лучше уточняют значение характеристик случайной величины. 

Для нормального, равномерного распределения и распределения Лапласа эти значения схожи и близки к нулю. У распределение Пуассона среднее значение $E(z)$ во всех выборках близко к 10, это значение параметра задания данного распределения. В характеристиках распределения Коши появляются аномально большие значения, это может объясняться неопределенностью математического ожидания и бесконечностью дисперсии случайной величины, распределенной по закону Коши.
\end{document}