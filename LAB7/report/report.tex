\documentclass[12pt,a4paper]{article}

\usepackage[T1,T2A]{fontenc}
\usepackage[utf8]{inputenc}
\usepackage[english, russian]{babel}
\usepackage{indentfirst}
\usepackage{misccorr}
\usepackage{graphicx}
\usepackage{amsmath}
\usepackage{graphicx}
\usepackage{float}
\usepackage[left=20mm,right=10mm, top=20mm,bottom=20mm,bindingoffset=0mm]{geometry}

\setlength{\parskip}{6pt}\graphicspath{{images/}}\DeclareGraphicsExtensions{.png}

\begin{document}

    \begin{titlepage}
        \begin{center}
            \large
            Санкт-Петербургский политехнический университет\\Петра Великого\\
            \vspace{0.5cm}
            Институт прикладной математики и механики\\
            \vspace{0.25cm}
            Кафедра «Прикладная математика»
            \vfill
            \textsc{\LARGE\textbf{Отчет по лабораторной работе №7}}\\[5mm]
            \Large
            по дисциплине\\"Математическая статистика"
        \end{center}
        \vfill
        \begin{tabular}{l p{175pt} l}
            Выполнил студент \\ группы 3630102/80201 && Хрипунков Дмитрий Викторович
            \vspace{0.25cm}
            \\Проверил \\ доцент, к.ф.-м.н. && Баженов Александр Николаевич
        \end{tabular}
        \vfill
        \begin{center}
            Санкт-Петербург \\ 2021 г.
        \end{center}
    \end{titlepage}
    
\newpage
\begin{center}
    \tableofcontents
    \setcounter{page}{2}
\end{center}
\newpage
\begin{center}
    \listoftables
\end{center}

\newpage
\section{Постановка задачи}
\begin{enumerate}
    \item Сгенерировать выборку объемом 100 элементов для нормального распределения $N(x,0,1)$
    \item По сгенерированной выборке оценить параметры $\mu$ и $\sigma$ нормального закона методом максимального правдоподобия. В качестве основной гипотезы $H_0$ будем считать, что сгенерированное распределение имеет вид $N(x,\hat\mu,\hat\sigma)$
    \item Проверить основную гипотезу, используя критерий согласия $\chi^2$. В качестве уровня значимости взять $\alpha=0.05$
    \item Привести таблицу вычислений $\chi^2$
    \item Исследовать точность (чувствительность) критерия $\chi^2$ - сгенерировать выборки равномерного распределения и распределения Лапласа малого объема (например, 20 элементов). Проверить их на нормальность
\end{enumerate}

\section{Теория}
\subsection{Метод максимального правдоподобия}
Пусть $x_1,...,x_n$ — случайная выборка из генеральной совокупности с плотностью вероятности $f(x,\theta)$. $L(x_1,...,x_n,\theta)$ — функция правдоподобия (ФП), рассматриваемая как функция неизвестного параметра $\theta$:
\begin{equation}
    L(x_1,...,x_n,\theta)=f(x_1,\theta)f(x_2,\theta)...f(x_n,\theta)
\end{equation}

\textit{Оценка максимального правдоподобия}:
\begin{equation}
    \hat{\theta}_{\textup{мп}}=\arg \max_\theta L(x_1,...,x_n,\theta)
\end{equation}

Система уравнений правдоподобия (в случае дифференцируемости функции правдоподобия):
\begin{equation}
    \frac{\partial\ln{L}}{\partial\theta} = 0 \hspace{8pt} \textup{или} \hspace{8pt} \frac{\partial\ln{L}}{\partial\theta} = 0, \hspace{20pt} k = 1, ..., m
\end{equation}

\subsection{Проверка гипотезы о законе распределения генеральной совокупности. Метод хи-квадрат}
Выдвинута гипотеза $H_0$ о генеральном закона распределения с функцией распределения $F(x)$. Необходимо оценить его параметры и проверить закон в целом.

Для проверки гипотезы о законе распределения чаще всего применяется критерий согласия $\chi^2$. Пусть гипотетическая функция распределения $F(x)$ не содержит неизвестных параметров.

Разобьём генеральную совокупность, т.е. множество значений изучаемой случайной величины $X$ на $k$ непересекающихся равных подмножеств $\Delta_1,\Delta_2,...,\Delta_k$, где $k$ выбирается согласованным с $n$ и берется аналогичному при построении гистограмм $k\approx1.72\sqrt[3]{n}$ или по формуле Старджесса $k\approx1+3.3\lg{n}$.

Пусть $p_i=P(X\in\Delta_i),i=\overline{1,k}$. Если генеральная совокупность - вся вещественная ось, $p_i=F(a_i)-F(a_{i-1}),i=\overline{1,k}$. При этом $\sum_{i=1}^k{p_i}=1$ и $p_i>0,i=\overline{1,k}$.

Пусть, далее, $n_1,n_2,..,n_k$ — частоты попадания выборочных элементов в подмножества $\Delta_1,\Delta_2,...,\Delta_k$ соответственно.

Если гипотеза $H_0$ справедлива, то относительные частоты $\frac{n_i}{n}\rightarrow p_i, i=\overline{1,k}$. Следовательно, мера отклонения выборочного распределения от гипотетического с использованием коэффициентов Пирсона:
\begin{equation}
    \chi^2=\sum_{i=1}^k{\frac{{n_i-np_i}^2}{np_i}}
\end{equation}

\textit{Теорема К.Пирсона:} статистика критерия $\chi^2$ асимптотически распределена по закону $\chi^2$ с $k-1$ степенями свободы.

\textbf{Правило проверки гипотезы о законе распределения по методу $\chi^2$}
\begin{enumerate}
    \item Выбираем уровень значимости $\alpha$
    \item Находим квантиль $\chi^2_{1-\alpha}(k - 1)$ распределения хи-квадрат с $k-1$ степенями свободы порядка $1-\alpha$
    \item С помощью гипотетической функции распределения $F(x)$ вычисляем вероятности $p_i=P(X\in\Delta_i),i=\overline{1,k}$
    \item Находим частоты $n_i$ попадания элементов выборки в подмножества $\Delta_i,i=\overline{1,k}$
    \item Вычисляем выборочное значение статистики критерия $\chi^2$
    \item Сравниваем $\chi^2_B$ и квантиль $\chi^2_{1-\alpha}(k-1):$
        \begin{itemize}
            \item если $\chi^2_B<\chi^2_{1-\alpha}(k-1)$, то гипотеза $H_0$ на данном этапе проверки принимается
            \item иначе гипотеза $H_0$ отвергается, выбирается одно из альтернативных распределений, и процедура проверки повторяется
        \end{itemize}
\end{enumerate}

\textit{Замечание:} при ситуации $\chi^2_B\approx\chi^2_{1-\alpha}(k-1)$ стоит увеличить объем выборки (например, в 2 раза), чтобы требуемое неравенство было более четким.

\textit{Замечание:} Изучено, что если для каких-либо подмножеств $\Delta_i,i=\overline{1,k}$ условие $np_i\geq5$ не выполняется, то следует объединить соседние подмножества (промежутки). Это условие выдвигается требованием близости величин $\frac{(n_i-np_i)}{\sqrt{np_i}}$. Тогда случайная величина будет распределена по закону, близкому к хи-квадрат. Такая близость обеспечивается достаточной численностью элементов в подмножествах $\Delta_i$.


\section{Реализация}
Лабораторная работа выполнена на языке Python в виртуальной среде Anaconda с интерпретатором версии 3.9 в среде разработки Visual Studio Code. Дополнительные зависимости:
\begin{itemize}
    \item scipy
    \item numpy
    \item tabulate
\end{itemize}

Исходный код размещён в git-репозитории на GitHub: \\ https://github.com/ThinkingFrog/MathStat

\section {Результаты}
\subsection{Выборка нормального распределения}
\begin{equation}
    \left\{
    \begin{array}{ll}
        \hat{\mu} = 0.02290129\\
        \hat{\sigma} = 1.08061897\\
        k = 8\\
        \alpha = 0.05\\
        \chi^2_{1 - \alpha}(k - 1)=\chi^2_{0.95} \approx 14.06714045
    \end{array}
    \right.
\end{equation}

\begin{table}[H]
    \centering
    \begin{tabular}{|c|c|c|c|c|c|c|}
    \hline
    $i$ & Границы $\Delta_i$ & $n_i$ & $p_i$ & $np_i$ & $n_i-np_i$ & $\frac{(n_i-np_i)^2}{np_i}$ \\ \hline
     1 & [-inf, -1.1]               &  17 & 0.135666  &  13.5666  &  3.43339 & 0.868913 \\ \hline
     2 & [-1.1, -0.73333333]        &   5 & 0.0960115 &   9.60115 & -4.60115 & 2.20501  \\ \hline
     3 & [-0.73333333, -0.36666667] &  16 & 0.125256  &  12.5256  &  3.47437 & 0.963726 \\ \hline
     4 & [-0.36666667, 0.0]         &  12 & 0.143066  &  14.3066  & -2.30662 & 0.371889 \\ \hline
     5 & [0.0, 0.36666667]          &  16 & 0.143066  &  14.3066  &  1.69338 & 0.200435 \\ \hline
     6 & [0.36666667, 0.73333333]   &   8 & 0.125256  &  12.5256  & -4.52563 & 1.63515  \\ \hline
     7 & [0.73333333, 1.1]          &   8 & 0.0960115 &   9.60115 & -1.60115 & 0.267019 \\ \hline
     8 & [1.1, inf]                 &  18 & 0.135666  &  13.5666  &  4.43339 & 1.44878  \\ \hline
     $\sum$ & -                          & 100 & 1         & 100       &  0       & 7.96092  \\ \hline
    \end{tabular} 
    \caption{$\chi^2_B$ при нормальном распределении}
\end{table}

\subsection{Выборка распределения Лапласа}
\begin{equation}
    \left\{
    \begin{array}{ll}
        \hat{\mu} = 0.01754504\\
        \hat{\sigma} = 0.94509717\\
        k = 5\\
        \alpha = 0.05\\
        \chi^2_{1 - \alpha}(k - 1)=\chi^2_{0.95} \approx 9.48772903
    \end{array}
    \right.
\end{equation}

\begin{table}[H]
    \centering
    \begin{tabular}{|c|c|c|c|c|c|c|}
    \hline
    $i$ & Границы $\Delta_i$ & $n_i$ & $p_i$ & $np_i$ & $n_i-np_i$ & $\frac{(n_i-np_i)^2}{np_i}$ \\ \hline
    1 & [-inf, -1.1]              &  4 & 0.135666 &  2.71332 &  1.28668  & 0.610153  \\ \hline
    2 & [-1.1, -0.36666667]       &  2 & 0.221268 &  4.42536 & -2.42536  & 1.32924   \\ \hline
    3 & [-0.36666667, 0.36666667] &  7 & 0.286132 &  5.72265 &  1.27735  & 0.285118  \\ \hline
    4 & [0.36666667, 1.1]         &  4 & 0.221268 &  4.42536 & -0.425356 & 0.0408842 \\ \hline
    5 & [1.1, inf]                &  3 & 0.135666 &  2.71332 &  0.286679 & 0.0302893 \\ \hline
    $\sum$ & -                         & 20 & 1        & 20       &  0        & 2.29568   \\  \hline
    \end{tabular} 
    \caption{$\chi^2_B$ при распределении Лапласа}
\end{table}

\subsection{Выборка равномерного распределения}
\begin{equation}
    \left\{
    \begin{array}{ll}
        \hat{\mu} = -0.08435758\\
        \hat{\sigma} = 0.95734159\\
        k = 5\\
        \alpha = 0.05\\
        \chi^2_{1 - \alpha}(k - 1)=\chi^2_{0.95} \approx 9.48772904
    \end{array}
    \right.
\end{equation}

\begin{table}[H]
    \centering
    \begin{tabular}{|c|c|c|c|c|c|c|}
    \hline
    $i$ & Границы $\Delta_i$ & $n_i$ & $p_i$ & $np_i$ & $n_i-np_i$ & $\frac{(n_i-np_i)^2}{np_i}$ \\ \hline
    1 & [-inf, -1.1]              &  3 & 0.135666 &  2.71332 &  0.286679 & 0.0302893 \\ \hline
    2 & [-1.1, -0.36666667]       &  4 & 0.221268 &  4.42536 & -0.425356 & 0.0408842 \\ \hline
    3 & [-0.36666667, 0.36666667] &  7 & 0.286132 &  5.72265 &  1.27735  & 0.285118  \\ \hline
    4 & [0.36666667, 1.1]         &  4 & 0.221268 &  4.42536 & -0.425356 & 0.0408842 \\ \hline
    5 & [1.1, inf]                &  2 & 0.135666 &  2.71332 & -0.713321 & 0.187529  \\ \hline
    $\sum$ & -                         & 20 & 1        & 20       &  0        & 0.584706  \\ \hline
    \end{tabular} 
    \caption{$\chi^2_B$ при равномерном распределении}
\end{table}

\section{Обсуждение}
Из приведённых таблиц видно, что для всех трёх распределений $\chi^2_B<\chi^2_{0.95}$.

Для выборки, распределенной по закону $N(x,\hat{\mu},\hat{\sigma})$, гипотеза $H_0$ о нормальном распределении с уровнем значимости $\alpha=0.05$ согласуется с определенной точностью. Для распределения Лапласа и равномерного распределения выборка меньшей размерности, из-за чего эта оценка хуже приближается к параметрам нормального распределения.

Для всех трёх случаев гипотеза $H_0$ принята.
\end{document}