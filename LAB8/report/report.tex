\documentclass[12pt,a4paper]{article}

\usepackage[T1,T2A]{fontenc}
\usepackage[utf8]{inputenc}
\usepackage[english, russian]{babel}
\usepackage{indentfirst}
\usepackage{misccorr}
\usepackage{graphicx}
\usepackage{amsmath}
\usepackage{graphicx}
\usepackage{float}
\usepackage[left=20mm,right=10mm, top=20mm,bottom=20mm,bindingoffset=0mm]{geometry}

\setlength{\parskip}{6pt}\graphicspath{{images/}}\DeclareGraphicsExtensions{.png}

\begin{document}

    \begin{titlepage}
        \begin{center}
            \large
            Санкт-Петербургский политехнический университет\\Петра Великого\\
            \vspace{0.5cm}
            Институт прикладной математики и механики\\
            \vspace{0.25cm}
            Кафедра «Прикладная математика»
            \vfill
            \textsc{\LARGE\textbf{Отчет по лабораторной работе №8}}\\[5mm]
            \Large
            по дисциплине\\"Математическая статистика"
        \end{center}
        \vfill
        \begin{tabular}{l p{175pt} l}
            Выполнил студент \\ группы 3630102/80201 && Хрипунков Дмитрий Викторович
            \vspace{0.25cm}
            \\Проверил \\ доцент, к.ф.-м.н. && Баженов Александр Николаевич
        \end{tabular}
        \vfill
        \begin{center}
            Санкт-Петербург \\ 2021 г.
        \end{center}
    \end{titlepage}
    
\newpage
\begin{center}
    \tableofcontents
    \setcounter{page}{2}
\end{center}
\newpage
\begin{center}
    \listoffigures
\end{center}

\newpage
\section{Постановка задачи}
\begin{enumerate}
    \item Провести дисперсионный анализ с применением критерия Фишера по данным регистраторов для одного сигнала
    \item Определить области однородности сигнала, переходные области, шум/фон
    \item Длину сигнала считать равной 1024
\end{enumerate}

\section{Теория}
\subsection{Величины дисперсионного анализа}
Необходимо вычислить следующие величины:
\begin{enumerate}
    \item \textit{Внутригрупповая дисперсия}
        \begin{equation}
            s_{IntraGroup}^2=\frac{1}{k}\sum_{i=1}^ks_i^2=\frac{1}{k}\sum_{i=1}^k\frac{\sum_{j=1}^n(x_{ij}-X_{cp})^2}{k-1},
        \end{equation}
        где $X_{cp}$ - среднее для части выборки, $k$ - количество частей выборки, $n$ - количество элементов в рассматриваемой части выборки.\\Внутригрупповая дисперсия является дисперсией совокупности и рассматривается как среднее значение выборочных дисперсий.
    \item \textit{Межгрупповая дисперсия}
        \begin{equation}
            s_{InterGroup}^2=k\frac{\sum_{i=1}^k(X_{i_{cp}}-X_{cp})^2}{k-1},
        \end{equation}
        где $X_{1_{cp}},X_{2_{cp}},...,X_{k_{cp}}$ - среднее значение для подвыборок, а $X_{cp}$ - среднее значение этих cредних значений подвыборок
    \item \textit{Значение критерия Фишера}
        \begin{equation}
            F=\frac{s_{InterGroup}^2}{s_{IntraGroup}^2}
        \end{equation}
\end{enumerate}

\section{Ход работы}
Лабораторная работа выполнена на языке Python в виртуальной среде Anaconda с интерпретатором версии 3.9 в среде разработки Visual Studio Code. Дополнительные зависимости:
\begin{itemize}
    \item numpy
    \item matplotlib
\end{itemize}

Исходный код размещён в git-репозитории на GitHub: \\ https://github.com/ThinkingFrog/MathStat

Алгоритм:
\begin{enumerate}
    \item Извлечь сигнал из исходных данный в файле \textit{wave\_ampl.txt}. Из условия известно, что сигнал имеет длину 1024, поэтому необходимо выбрать начальный индекс, кратный длине сигнала
    \item Построить гистограмму с соответствующими столбцами:
        \begin{itemize}
            \item Фон - столбец с наибольшим значением
            \item Переходы - столбцы с малыми значениями
            \item Сигнал - второй по величине столбец после фона
        \end{itemize}
    \item Устранить явные выбросы, для этого был использован медианный фильтр (выброс = среднее арифметическое его соседей). Получается сглаженный сигнал
    \item Разделить сигнал на области: сигнал, фон и переходные процессы
    \item Определить тип области по критерию Фишера:
        \begin{itemize}
            \item Если значение критерия Фишера велико, это будут переходные процессы
            \item Если значение критерия Фишера находится вблизи 1, эти области однородны
        \end{itemize}
\end{enumerate}

\section {Результаты}
Рассматривается сигнал с номером 74
\begin{figure}[H]
    \centering
    \includegraphics[scale=0.8]{Signal.png}
    \caption{График входного сигнала}
\end{figure}
\begin{figure}[H]
    \centering
    \includegraphics[scale=0.8]{Signal_hist.png}
    \caption{Гистограмма входного сигнала}
\end{figure}
\begin{figure}[H]
    \centering
    \includegraphics[scale=0.8]{Zones.png}
    \caption{Области сигнала}
\end{figure}
\begin{table}[H]
    \centering
    \begin{tabular}{|c|c|c|c|}
        \hline
        Область & Тип & Количество разбиений & Критерий Фишера\\\hline
        [0, 239] & Фон & 4 & 0.07777439\\\hline
        [239, 307] & Переход & 4 & 17.37498936\\\hline
        [307, 732] & Сигнал & 5 & 0.236059\\\hline
        [732, 800] & Переход & 4 & 17.16207435\\\hline
        [800, 1023] & Фон & 4 & 0.56926051\\\hline
    \end{tabular}
    \caption{Характеристика выделенных областей}
\end{table}

\section{Обсуждение}
Данные были разделены на 5 частей: фон, переход, сигнал, переход, фон. Значение критерия Фишера для обеих областей фона и области сигнала близки к 1, поэтому эти области однородны. При этом обе области перехода неоднородны, так как имеют значения критерия Фишера много больше 1.

График сигнала симметричен по вертикальной оси относительно центра и не содержит заметных выбросов.
\end{document}