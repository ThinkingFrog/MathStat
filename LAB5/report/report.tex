\documentclass[12pt,a4paper]{article}

\usepackage[T1,T2A]{fontenc}
\usepackage[utf8]{inputenc}
\usepackage[english, russian]{babel}
\usepackage{indentfirst}
\usepackage{misccorr}
\usepackage{graphicx}
\usepackage{amsmath}
\usepackage{graphicx}
\usepackage{float}
\usepackage[left=20mm,right=10mm, top=20mm,bottom=20mm,bindingoffset=0mm]{geometry}

\setlength{\parskip}{6pt}\graphicspath{{images/}}\DeclareGraphicsExtensions{.png}

\begin{document}

    \begin{titlepage}
        \begin{center}
            \large
            Санкт-Петербургский политехнический университет\\Петра Великого\\
            \vspace{0.5cm}
            Институт прикладной математики и механики\\
            \vspace{0.25cm}
            Кафедра «Прикладная математика»
            \vfill
            \textsc{\LARGE\textbf{Отчет по лабораторной работе №5}}\\[5mm]
            \Large
            по дисциплине\\"Математическая статистика"
        \end{center}
        \vfill
        \begin{tabular}{l p{175pt} l}
            Выполнил студент \\ группы 3630102/80201 && Хрипунков Дмитрий Викторович
            \vspace{0.25cm}
            \\Проверил \\ доцент, к.ф.-м.н. && Баженов Александр Николаевич
        \end{tabular}
        \vfill
        \begin{center}
            Санкт-Петербург \\ 2021 г.
        \end{center}
    \end{titlepage}
    
\newpage
\begin{center}
    \tableofcontents
    \setcounter{page}{2}
\end{center}
\newpage
\begin{center}
    \listoffigures
\end{center}

\newpage
\section{Постановка задачи}
\begin{enumerate}
    \item Сгенерировать двумерные выборки размерами 20, 60 и 100 элементов для нормального двумерного распределения $N(x, y, 0, 0, 1, 1, \rho)$ с коэффициентом корреляции $\rho$, равным 0, 0.5, 0.9
    \item Сгенерировать каждую выборку 1000 раз и вычислить среднее значение, среднее значение квадрата и дисперсию коэффициентов корреляции Пирсона, Спирмена и квадрантного коэффициента корреляции
    \item Повторить все вычисления для смеси нормальных распределений:
        \begin{equation}
	        f(x,y)=0.9N(x,y,0,0,1,1,0.9)+0.1N(x,y,0,0,10,10,-0.9)
        \end{equation}
    \item Изобразить сгенерированные точки на плоскости и нарисовать эллипс равновероятности
\end{enumerate}

\section{Теория}
\subsection{Двумерное нормальное распределение}
Двумерная случайная величина $(X,Y)$ называется распределённой нормально (или просто нормальной), если её плотность вероятности определена формулой
\begin{equation}
    N(x,y,\bar{x},\bar{y},\sigma_x,\sigma_y,\rho)=\frac{1}{2\pi\sigma_x\sigma_y\sqrt{1-\rho^2}}\times\exp{
    \begin{Bmatrix}
        -\frac{1}{2(1-\rho^2)}
        \begin{bmatrix}
            \frac{(x-\bar{x})^2}{\sigma_x^2}-2\rho\frac{(x-\bar{x})(y-\bar{y})}{\sigma_x\sigma_y}+\frac{(y-\bar{y})^2}{\sigma_y^2}
        \end{bmatrix}
    \end{Bmatrix}
    }
\end{equation}

Компоненты $X,Y$ двумерной нормальной случайной величины также распределены нормально с математическими ожиданиями $\bar{x}$,$\bar{y}$ и средними квадратическими отклонениями $\sigma_{x},\sigma_{y}$ соответственно.

Параметр $\rho$ называется коэффициентом корреляции.

\subsection{Корреляционный момент (ковариация) и коэффициент корреляции}
\textit{Корреляционный момент (ковариация)} двух случайных величин $X$ и $Y$:
\begin{equation}
    K=cov(X,Y)=M[(X-\bar{x})(Y-\bar{y})]
\end{equation}

\textit{Коэффициент корреляции $\rho$} двух случайных величин $X$ и $Y$:
\begin{equation}
    \rho=\frac{K}{\sigma_x\sigma_y}
\end{equation}

\subsection{Выборочные коэффициенты корреляции}
\subsubsection{Выборочный коэффициент корреляции Пирсона}
Выборочный коэффициент корреляции Пирсона:
\begin{equation}
    r=\frac{\frac{1}{n}\sum{(x_i-\bar{x})(y_i-\bar{y})}}{\sqrt{\frac{1}{n}\sum{(x_i-\bar{x})^2}\frac{1}{n}\sum{(y_i-\bar{y})^2}}}=\frac{K}{s_Xs_Y},
\end{equation}
где $K,s^2_X,s^2_Y$ — выборочные ковариация и дисперсии с.в. $X$ и $Y$

\subsubsection{Выборочный квадрантный коэффициент корреляции}
Выборочный квадрантный коэффициент корреляции:
\begin{equation}
    r_Q=\frac{(n_1+n_3)-(n_2+n_4)}{n},
\end{equation}
где $n_1,n_2,n_3,n_4$ - количества точек с координатами $x_i,y_i$, попавшими соответственно в I, II, III, IV квадранты декартовой системы с осями $x'=x-medx,y'=y-medy$ и с центром в точке с координатами $(medx,medy)$

\subsubsection{Выборочный коэффициент ранговой корреляции Спирмена}
Обозначим ранги, соответствующие значениям переменной $X$, через $u$, а ранги, соответствующие значениям переменной $Y$, — через $v$.

Выборочный коэффициент ранговой корреляции Спирмена:
\begin{equation}
    r_S=\frac{\frac{1}{n}\sum{(u_i-\bar{u})(v_i-\bar{v})}}{\sqrt{\frac{1}{n}\sum{(u_i-\bar{u})^2}\frac{1}{n}\sum{(v_i-\bar{v})^2}}},
\end{equation}
где $\bar{u}=\bar{v}=\frac{1+2+...+n}{n}=\frac{n+1}{2}$ — среднее значение рангов

\subsection{Эллипсы рассеивания}
Уравнение проекции эллипса рассеивания на плоскость $xOy$:
\begin{equation}
    \frac{(x-\bar{x})^2}{\sigma_x^2}-2\rho\frac{(x-\bar{x})(y-\bar{y})}{\sigma_x\sigma_y}+\frac{(y-\bar{y})^2}{\sigma_y^2}=const
\end{equation}

Центр эллипса находится в точке с координатами $(\bar{x},\bar{y})$, оси симметрии эллипса составляют с осью $Ox$ углы, определяемые уравнением:
\begin{equation}
    \tg(2\alpha)=\frac{2\rho\sigma_x\sigma_y}{\sigma_x^2-\sigma_y^2}
\end{equation}

\section{Реализация}
Лабораторная работа выполнена на языке Python в виртуальной среде Anaconda с интерпретатором версии 3.9 в среде разработки Visual Studio Code. Дополнительные зависимости:
\begin{itemize}
    \item scipy
    \item numpy
    \item matplotlib
\end{itemize}

Исходный код размещён в git-репозитории на GitHub: \\ https://github.com/ThinkingFrog/MathStat

\section {Результаты}
\subsection{Вычислительные характеристики распределения}
\begin{table}[H]
    \centering
    \begin{tabular}{|c|c|c|c|}
        \hline
        & $r$ & $r_S$ & $r_Q$\\\hline
        $N = 20, \rho = 0$ & & &\\\hline
        $E(z)$ & 0.4904 & 0.4684 & 0.4\\\hline
        $E(z^2)$ & 0.2405 & 0.2194 & 0.16\\\hline
        $D(z)$ & 0.0413 & 0.0392 & 0.051\\\hline
        \hline
        $N = 20, \rho = 0.5$ & & &\\\hline
        $E(z)$ & 0.7963 & 0.7692 & 0.6\\\hline
        $E(z^2)$ & 0.6341 & 0.5916 & 0.36\\\hline
        $D(z)$ & 0.0097 & 0.0134 & 0.0378\\\hline
        \hline
        $N = 20, \rho = 0.9$ & & &\\\hline
        $E(z)$ & 0.965 & 0.9489 & 0.8\\\hline
        $E(z^2)$ & 0.9313 & 0.9004 & 0.64\\\hline
        $D(z)$ & 0.0004 & 0.0011 & 0.0204\\\hline
        \hline
        $N = 60, \rho = 0$ & & &\\\hline
        $E(z)$ & 0.4796 & 0.4705 & 0.3333\\\hline
        $E(z^2)$ & 0.23 & 0.2214 & 0.1111\\\hline
        $D(z)$ & 0.0117 & 0.0115 & 0.0156\\\hline
        \hline
        $N = 60, \rho = 0.5$ & & &\\\hline
        $E(z)$ & 0.7845 & 0.7705 & 0.6\\\hline
        $E(z^2)$ & 0.6155 & 0.5936 & 0.36\\\hline
        $D(z)$ & 0.0031 & 0.0039 & 0.0118\\\hline
        \hline
        $N = 60, \rho = 0.9$ & & &\\\hline
        $E(z)$ & 0.9632 & 0.955 & 0.8\\\hline
        $E(z^2)$ & 0.9277 & 0.9119 & 0.64\\\hline
        $D(z)$ & 0.0001 & 0.0003 & 0.0056\\\hline
        \hline
        $N = 100, \rho = 0$ & & &\\\hline
        $E(z)$ & 0.4628 & 0.4627 & 0.32\\\hline
        $E(z^2)$ & 0.2141 & 0.2141 & 0.1024\\\hline
        $D(z)$ & 0.0072 & 0.0072 & 0.01\\\hline
        \hline
        $N = 100, \rho = 0.5$ & & &\\\hline
        $E(z)$ & 0.7868 & 0.7711 & 0.56\\\hline
        $E(z^2)$ & 0.6191 & 0.5946 & 0.3136\\\hline
        $D(z)$ & 0.002 & 0.0024 & 0.0067\\\hline
        \hline
        $N = 100, \rho = 0.9$ & & &\\\hline
        $E(z)$ & 0.9625 & 0.9565 & 0.84\\\hline
        $E(z^2)$ & 0.9263 & 0.9149 & 0.7056\\\hline
        $D(z)$ & 0.0001 & 0.0001 & 0.0034\\\hline
    \end{tabular}
    \caption{Характеристики нормального распределения}
\end{table}

\begin{table}[H]
    \centering
    \begin{tabular}{|c|c|c|c|}
        \hline
        & $r$ & $r_S$ & $r_Q$\\\hline
        $N = 20$ & & &\\\hline
        $E(z)$ & 0.9249 & 0.9008 & 0.8\\\hline
        $E(z^2)$ & 0.8555 & 0.8114 & 0.64\\\hline
        $D(z)$ & 0.0024 & 0.0038 & 0.0257\\\hline
        \hline
        $N = 60$ & & &\\\hline
        $E(z)$ & 0.92 & 0.908 & 0.7333\\\hline
        $E(z^2)$ & 0.8464 & 0.8244 & 0.5378\\\hline
        $D(z)$ & 0.0006 & 0.0009 & 0.0076\\\hline
        \hline
        $N = 100$ & & &\\\hline
        $E(z)$ & 0.9191 & 0.9097 & 0.76\\\hline
        $E(z^2)$ & 0.8447 & 0.8276 & 0.5776\\\hline
        $D(z)$ & 0.0003 & 0.0005 & 0.0045\\\hline
    \end{tabular}
    \caption{Характеристики смеси нормальных распределений}
\end{table}

\subsection{Эллипсы рассеивания}
\begin{figure}[H]
    \centering
    \includegraphics[scale=0.85]{Ellipse_1.png}
    \caption{Эллипс рассеивания для выборки из 20 элементов}
\end{figure}

\begin{figure}[H]
    \centering
    \includegraphics[scale=0.85]{Ellipse_2.png}
    \caption{Эллипс рассеивания для выборки из 60 элементов}
\end{figure}

\begin{figure}[H]
    \centering
    \includegraphics[scale=0.85]{Ellipse_3.png}
    \caption{Эллипс рассеивания для выборки из 100 элементов}
\end{figure}

\section{Обсуждение}
По таблицам характеристик распределений можно заметить, что для характеристик $E(z),E(z^2)$ в большинстве случаев справедливо соотношение $r > r_S > r_Q$, тогда как для $D(z)$ выполняется обратное соотношение $r_Q > r_S > r$. Также чем больше коэффициент корреляции $\rho$, тем ближе дисперсия становится к нулю.

На изображениях эллипсов рассеивания можно увидеть, что почти все элементы выборки находятся внутри эллипсов. Это соответствует теоретической оценке результата. Также большая часть значений концентрируется в центре эллипса, что соответствует его теоретической оценке.
\end{document}